%!TeX encoding = UTF-8
%!TeX program = xelatex
\documentclass[notheorems, aspectratio=54]{beamer}
% aspectratio: 1610, 149, 54, 43(default), 32
\usepackage{qrcode}
\usepackage{xcolor,lipsum}
\usepackage{latexsym}
\usepackage{amsmath,amssymb}
\usepackage{mathtools}
\usepackage{color,xcolor}
\usepackage{graphicx}
\usepackage{algorithm}
\usepackage{amsthm}
\usepackage{lmodern} % 解决 font warning
% \usepackage[UTF8]{ctex}
\usepackage{animate} % insert gif

\usepackage{lipsum} % To generate test text 
\usepackage{ulem}

\usepackage{listings} % display code on slides; don't forget [fragile] option after \begin{frame}

% ----------------------------------------------
% tikx
\usepackage{framed}
\usepackage{tikz}
\usepackage{pgf}
\usetikzlibrary{calc,trees,positioning,arrows,chains,shapes.geometric,%
    decorations.pathreplacing,decorations.pathmorphing,shapes,%
    matrix,shapes.symbols}
\pgfmathsetseed{1} % To have predictable results
% Define a background layer, in which the parchment shape is drawn
\pgfdeclarelayer{background}
\pgfsetlayers{background,main}

% define styles for the normal border and the torn border
\tikzset{
  normal border/.style={black!70!gray, decorate, 
     decoration={random steps, segment length=2.5cm, amplitude=.7mm}},
  torn border/.style={black!70!gray, decorate, 
     decoration={random steps, segment length=.5cm, amplitude=1.7mm}}}

% Macro to draw the shape behind the text, when it fits completly in the
% page
\def\parchmentframe#1{
\tikz{
  \node[inner sep=2em] (A) {#1};  % Draw the text of the node
  \begin{pgfonlayer}{background}  % Draw the shape behind
  \fill[normal border] 
        (A.south east) -- (A.south west) -- 
        (A.north west) -- (A.north east) -- cycle;
  \end{pgfonlayer}}}

% Macro to draw the shape, when the text will continue in next page
\def\parchmentframetop#1{
\tikz{
  \node[inner sep=2em] (A) {#1};    % Draw the text of the node
  \begin{pgfonlayer}{background}    
  \fill[normal border]              % Draw the ``complete shape'' behind
        (A.south east) -- (A.south west) -- 
        (A.north west) -- (A.north east) -- cycle;
  \fill[torn border]                % Add the torn lower border
        ($(A.south east)-(0,.2)$) -- ($(A.south west)-(0,.2)$) -- 
        ($(A.south west)+(0,.2)$) -- ($(A.south east)+(0,.2)$) -- cycle;
  \end{pgfonlayer}}}

% Macro to draw the shape, when the text continues from previous page
\def\parchmentframebottom#1{
\tikz{
  \node[inner sep=2em] (A) {#1};   % Draw the text of the node
  \begin{pgfonlayer}{background}   
  \fill[normal border]             % Draw the ``complete shape'' behind
        (A.south east) -- (A.south west) -- 
        (A.north west) -- (A.north east) -- cycle;
  \fill[torn border]               % Add the torn upper border
        ($(A.north east)-(0,.2)$) -- ($(A.north west)-(0,.2)$) -- 
        ($(A.north west)+(0,.2)$) -- ($(A.north east)+(0,.2)$) -- cycle;
  \end{pgfonlayer}}}

% Macro to draw the shape, when both the text continues from previous page
% and it will continue in next page
\def\parchmentframemiddle#1{
\tikz{
  \node[inner sep=2em] (A) {#1};   % Draw the text of the node
  \begin{pgfonlayer}{background}   
  \fill[normal border]             % Draw the ``complete shape'' behind
        (A.south east) -- (A.south west) -- 
        (A.north west) -- (A.north east) -- cycle;
  \fill[torn border]               % Add the torn lower border
        ($(A.south east)-(0,.2)$) -- ($(A.south west)-(0,.2)$) -- 
        ($(A.south west)+(0,.2)$) -- ($(A.south east)+(0,.2)$) -- cycle;
  \fill[torn border]               % Add the torn upper border
        ($(A.north east)-(0,.2)$) -- ($(A.north west)-(0,.2)$) -- 
        ($(A.north west)+(0,.2)$) -- ($(A.north east)+(0,.2)$) -- cycle;
  \end{pgfonlayer}}}

% Define the environment which puts the frame
% In this case, the environment also accepts an argument with an optional
% title (which defaults to ``Example'', which is typeset in a box overlaid
% on the top border
\newenvironment{parchment}[1][Example]{%
  \def\FrameCommand{\parchmentframe}%
  \def\FirstFrameCommand{\parchmentframetop}%
  \def\LastFrameCommand{\parchmentframebottom}%
  \def\MidFrameCommand{\parchmentframemiddle}%
  \vskip\baselineskip
  \MakeFramed {\FrameRestore}
  \noindent\tikz\node[inner sep=1ex, draw=black!20, fill=black!90, 
          anchor=west, overlay] at (0em, 2em) {\sffamily#1};\par}%
{\endMakeFramed}

% ----------------------------------------------

\mode<presentation>{
    \usetheme{Warsaw}
    % Boadilla CambridgeUS
    % default Antibes Berlin Copenhagen
    % Madrid Montpelier Ilmenau Malmoe
    % Berkeley Singapore Warsaw
    \usecolortheme{seagull}
    % beetle, beaver, orchid, whale, dolphin, seagull
    \useoutertheme{infolines}
    % infolines miniframes shadow sidebar smoothbars smoothtree split tree
    \useinnertheme{circles}
    % circles, rectanges, rounded, inmargin
}

% ---------------------------------------------------------------------
% Jet Black Theme
\setbeamercolor{normal text}{fg=white,bg=black!90}
\setbeamercolor{structure}{fg=white}

\setbeamercolor{alerted text}{fg=red!85!black}

\setbeamercolor{item projected}{use=item,fg=black,bg=item.fg!35}

\setbeamercolor*{palette primary}{use=structure,fg=structure.fg}
\setbeamercolor*{palette secondary}{use=structure,fg=structure.fg!95!black}
\setbeamercolor*{palette tertiary}{use=structure,fg=structure.fg!90!black}
\setbeamercolor*{palette quaternary}{use=structure,fg=structure.fg!95!black,bg=black!80}

\setbeamercolor*{framesubtitle}{fg=white}

\setbeamercolor*{block title}{parent=structure,bg=black!70!gray}
\setbeamercolor*{block body}{fg=black,bg=black!10}
\setbeamercolor*{block title alerted}{parent=alerted text,bg=black!15}
\setbeamercolor*{block title example}{parent=example text,bg=black!15}
% ---------------------------------------------------------------------


% ---------------------------------------------------------------------
% flow chart
\tikzset{
    >=stealth',
    punktchain/.style={
        rectangle, 
        rounded corners, 
        % fill=black!10,
        draw=white, very thick,
        text width=6em,
        minimum height=2em, 
        text centered, 
        on chain
    },
    largepunktchain/.style={
        rectangle,
        rounded corners,
        draw=white, very thick,
        text width=10em,
        minimum height=2em,
        on chain
    },
    line/.style={draw, thick, <-},
    element/.style={
        tape,
        top color=white,
        bottom color=blue!50!black!60!,
        minimum width=6em,
        draw=blue!40!black!90, very thick,
        text width=6em, 
        minimum height=2em, 
        text centered, 
        on chain
    },
    every join/.style={->, thick,shorten >=1pt},
    decoration={brace},
    tuborg/.style={decorate},
    tubnode/.style={midway, right=2pt},
    font={\fontsize{10pt}{12}\selectfont},
}
% ---------------------------------------------------------------------

% code setting
\lstset{
    language=C++,
    basicstyle=\ttfamily\footnotesize,
    keywordstyle=\color{red},
    breaklines=true,
    xleftmargin=2em,
    numbers=left,
    numberstyle=\color[RGB]{222,155,81},
    frame=leftline,
    tabsize=4,
    breakatwhitespace=false,
    showspaces=false,               
    showstringspaces=false,
    showtabs=false,
    morekeywords={Str, Num, List},
}

% ---------------------------------------------------------------------

\newcommand{\reditem}[1]{\setbeamercolor{item}{fg=red}\item #1}

\newcommand*{\Scale}[2][4]{\scalebox{#1}{\ensuremath{#2}}}

\renewcommand\textbullet{\ensuremath{\bullet}}

% -------------------------------------------------------------

%% preamble
\title[Distributed Nash Equilibrium]{Distributed Nash Equilibrium Seeking On a Consensus based Gaming}
% \subtitle{The subtitle}
\author[Ryan Kung]{\includegraphics[height=2cm]{./ieee-blockchain.png} \\ Ryan J. Kung}

\institute[IEEE Blockchain]{ryankung@ieee.org}
% -------------------------------------------------------------
\begin{document}

% title frame
\begin{frame}
    \titlepage
\end{frame}

% normal frame
\section{Introduction}

\begin{frame}
  \frametitle{Introduction}
  \begin{itemize}
    \item Game
    \item Nash Equilibrium
    \item Distributed Game
  \end{itemize}

\end{frame}

\begin{frame}
  \frametitle{Game}
  Consider a game with $N$ players.
  The set of players is denoted by $\mathbb{N}={1, 2, ..., N}$.
  The payoff function of player $i$ is $f_i(x)$,
  where $x = [x_1, x_2,...,x_N]^T \in R^N$ is the vector of players' actions
  and $x_i \in R$ is the action of player $i$,
  then player $i$ has no direct access to player $j$'s action.

\end{frame}


\begin{frame}
  \frametitle{Nash Equilibrium}
  \begin{center}
    \bfseries{Nash equilibrium is an action profile on which
      no player can gain more payoff by unilaterally changing its own action.}
  \end{center}
  \begin{gather}
    \forall i, x_i \in S_i: f_i(x^*_i, x^*_{-i}) \geq f_i(x_i, x^*_{-i}) \nonumber
  \end{gather}
\end{frame}

\begin{frame}
  \frametitle{Cournot's duopoly Model}
  \begin{itemize}
  \item the players are the firms
  \item the actions of each firm are the set of possible outputs $Q$
  \item the payoff of each firm is its profit.
  \end{itemize}
  where:\newline \\
  \begin{tabular}{c    l}
    p:= & \text{Price, Inverse demand function}\\
    $u_i$:= & \text{Profit of player i} \\
    $c_i$:= & \text{Total Cost Function}\\\\
  \end{tabular}\nonumber
 
\end{frame}

\begin{frame}
  \frametitle{Cournot's duopoly Model}
    \begin{tabular}{c    l}
    p:= & \text{Price, Inverse demand function}\\
    $u_i$:= & \text{Profit of player i} \\
    $c_i$:= & \text{Total Cost Function}\\\\
  \end{tabular}\nonumber
   
    We have:\\
    \begin{gather}
      u_i(q_1,q_2)=q_ip(q)-c_i(q_i)
    \end{gather}
    Thus the response function $\dot{r}_i: Q_1\rightarrow Q_2$:\\
    \begin{gather}
      \dot{r}_1 = \dot{r}_2 = \frac{1-c}{3} \nonumber
    \end{gather}
\end{frame}




\begin{frame}
  \frametitle{Distributed}
  A system is distributed if the message transmission delay is not negligible compared to the time between events in a single process. \cite{time-clocks-ordering-events-distributed-system}\\

  \begin{figure}[H]
    \centering
    \includegraphics[width=0.5\linewidth]{lamportts.png}
    \caption{Lamport Timestamp}
  \end{figure}
\end{frame}

\begin{frame}
  \frametitle{Distributed Game}
  
    We defined Distributed Gaming as a series of Game Behaviors and Strategis which is Distributed. A Gaming is Distributed if the message transaction delay is not negligible compared to the time between event in classic gaming behavior.
  \end{frame}


\begin{frame}
  \frametitle{Formalize}
  Consider a game with $N$ players.
  The set of players is denoted by $\mathbb{N}={1, 2, ..., N}$.
  The payoff function of player $i$ is $f_i(x)$,
  where $x = [x_1, x_2,...,x_N]^T \in R^N$ is the vector of players' actions
  and $x_i \in R$ is the action of player $i$,
  then player $i$ has no direct access to player $j$'s action.\newline

  \bfseries{Suppose that if player $j$ is not a neighbor of player $i$, then player $i$ has no direct access to player $j's$ action}
\end{frame}





\section {Distributed Nash Equilibrium}


\begin{frame}
  \frametitle{Distributed Nash Equilibrium}
  \begin{itemize}
    \item Sensitive Game
    \item Distributed Network
  \end{itemize}

\end{frame}


\begin{frame}
  \frametitle{Example A}
  \begin{center}
    \begin{tabular}{c|c|c|}
      & L & R \\
      \hline
      U & (1,3)&(-3,0)\\
      \hline
      M & (-2,0)&(1,3)\\
      \hline
      D & (0,1)&(0,1)\\
    \end{tabular}
  \end{center}
  With Repeated Advantage Solution:
  \begin{center}
    \begin{tabular}{c|c|c|}
      & L \\
      \hline
      U & (1,3)\\
      \hline
    \end{tabular}
  \end{center}
  But when we discuss Mixed Strategy, if more than 1\% $players_2$ chose $R$, $D$ is better than $U$:
\end{frame}

\begin{frame}
  \frametitle{Example A :: Mixed Strategy}

  Mixed Strategy $\sigma$ is a probability distribution over pure strategy.\newline

  A Mixed Strategy for player $i$ can be present as a vector:
  \begin{gather}
    \left(\sigma_i(U), \sigma_i(M), \sigma_i(D)\right) \nonumber
  \end{gather}
  Payoff Function $u_i$ over $\sigma$:
   \begin{gather}
    \sum_{i \in S}\left (\prod_{j=i}^I \sigma_i (s_u)\right)u_j(s) \nonumber
  \end{gather}
 
\end{frame}



\begin{frame}
  \frametitle{Example A :: Mixed Strategy}
    \begin{center}
    \begin{tabular}{c|c|c|}
      & L & R \\
      \hline
      U & (1,3)&(-3,0)\\
      \hline
      M & (-2,0)&(1,3)\\
      \hline
      D & (0,1)&(0,1)\\
    \end{tabular}
  \end{center}
  Let:

  \begin{gather}
    \sigma_2 = (0.99, 0.01) \nonumber
  \end{gather}
  Then:
  \begin{gather}
    u_1 = -5.94 \iff \sigma_1 = (1, 0, 0)  \nonumber \\
    u_1 = 2  \iff \sigma_1 = (0, 0, 1) \nonumber
  \end{gather}

\end{frame}



\begin{frame}
  \frametitle{Example B}

  Prisoner's Dilemma
  \begin{center}
    \begin{tabular}{c|c|c|}
      & M & S \\
      \hline
      M & (1,1)&(-1, 2)\\
      \hline
      S & (2,-1)&(0,0)\\
    \end{tabular}
  \end{center}
  If players are sensitive on the uncertainty of $s_{-1}$, they may not choose the rational strategy.
\end{frame}

\begin{frame}
  \frametitle{Distributed Nash Equilibrium}

  The action of player $i$ is updated according to\newline
  \begin{gather}
    \dot{x} = k_i \frac {\partial f_i}{\partial x_i}(y_i), i \in \mathbb{N} 
  \end{gather}
  where
   \begin{gather}
     \dot{y}_i = [y_{i1}, y_{i2},...,y_{iN}]^T \in R \nonumber \\
     k_i = \delta \bar{k} \nonumber
   \end{gather}
   $\delta$ as a small positive parameter, $\bar{k}_i$ is a fixed positive parameter.\newline
   $y_{ij}, \forall i, j \in \mathbb{N}$ is player $i$'s estimate on player $j$'s action, which are generated by:\newline

   \begin{gather}
     \dot{y_{ij}}=-\left(\sum_{k=1}^N a_{ik}(y_{ik}-y_{kj})+a_{ij}(y_{ij}-x_y)\right) \nonumber
   \end{gather}

\end{frame}

\begin{frame}
  \frametitle{Distributed Nash Equilibrium}
  Let $\tau=\delta t$, at the $\tau$-time scale:\newline
  \begin{gather}
    \frac{dx_i}{d_{\tau}}=\bar{k_i}\frac{\partial f_i}{\partial x_i}(y_i)\nonumber\\
    \delta \frac{dy_{ij}}{d\tau}=-\left(\sum_{k=1}^N a_{ik}(y_{ik}-y_{kj})+a_{ij}(y_{ij}-x_y)\right)  \cite{7888532}
  \end{gather}
\end{frame}

\begin{frame}
  \frametitle{Distributed Nash Equilibrium}
  Easy to know, by setting $\delta$ to zero,  The reduced system is:
  \begin{gather}
    \frac{dx_i}{d\tau} = \bar{k}_i \frac{\partial f_i}{\partial x_i}(x) \ \  \forall i \in \mathbb{N}
  \end{gather}
  \begin{figure}[H]
    \centering
    \includegraphics[width=0.7\linewidth]{nash-eq.png}
  \end{figure}

\end{frame}
\begin{frame}
  \begin{figure}[H]
    \frametitle{Under Switching Topologies}
    \centering
    \includegraphics[height=0.4\linewidth]{nodes.png}
  \end{figure}
  Player $i$ can be regarded as a virtual leader, who provides its action as a reference signal to be followed by $-i$,   For each $X^*$, there is a constant $\delta$ that for every $\delta \in (\delta_{_max}, \delta_{min})$, Nash Equilibrium $X^*$ is asymptotically stable. \cite{8093754}
\end{frame}

\begin{frame}
  \frametitle{On loss of Communication}
    \begin{figure}[H]
    \centering
    \includegraphics[height=0.4\linewidth]{lose-com.png}
  \end{figure}
    Player $i$ can be regarded as a virtual leader, who provides its action as a reference signal to be followed by $-i$,   For each $X^*$, there is a constant $\delta$ that for every $\delta \in (\delta_{_max}, \delta_{min})$, Nash Equilibrium $X^*$ is asymptotically stable. \cite{8093754}
\end{frame}

\begin{frame}
  \frametitle{Conclusions}
    \begin{itemize}
    \item Exists Distributed Nash Equilibrium
    \item Distributed Nash Equilibrium can be sensitive
    \item Distributed Nash Equilibrium is stable under different network Topologies
    \end{itemize}
  \end{frame}

  
  \begin{frame}
    \frametitle{Reference}
  \bibliographystyle{plain}
  \bibliography{slide}
\end{frame}

\begin{frame}
  \frametitle{Thanks}
  Repository of this Slides:\newline
  - https://github.com/RyanKung/dneq4eos \quad
  \color{black}{\fcolorbox{white}{white}{\qrcode[link, height=0.5in]{https://github.com/RyanKung/dneq4eos}}}
\end{frame}
\end{document}